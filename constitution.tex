\documentclass[a4paper,fontsize=14pt,titlepage]{scrartcl}

\usepackage[T2A]{fontenc}
\usepackage[utf8]{inputenc}
\usepackage[russian]{babel}
\usepackage{indentfirst}
\usepackage{graphicx}
\usepackage[left=2cm,top=2cm,right=1.7cm,bottom=3cm]{geometry}
\usepackage[font={small}]{caption}
\usepackage{wrapfig}

\usepackage{enumitem}
\setlist{nolistsep}
%\setlist[1]{\labelindent=\parindent}
%\setlist[itemize]{leftmargin=*}
\setlist[itemize,1]{label=---}

\setkomafont{section}{\normalfont\bfseries}
\setkomafont{paragraph}{\normalfont}
\setkomafont{subparagraph}{\normalfont}
\setkomafont{title}{\rmfamily \huge}

\newcommand{\longpage}{\enlargethispage{\baselineskip}}
\newcommand{\shortpage}{\enlargethispage{-\baselineskip}}

% deep magic follows, don't touch it!
\makeatletter
\@newctr{paragraph}[section]
\@newctr{subparagraph}[paragraph]
\newenvironment{numberedpars}{%
  \addtocounter{secnumdepth}{1}
  \renewcommand\theparagraph{\arabic{section}.\arabic{paragraph}.}
  \renewcommand\@seccntformat[1]
  {\expandafter\ifx\csname##1\endcsname\paragraph\csname 
  the##1\endcsname\else\csname the##1\endcsname\fi}
  \def\paragraph{\par\addvspace{3.25ex \@plus1ex \@minus.2ex}{\raggedsection\normalfont\sectfont\nobreak\size@paragraph\refstepcounter{paragraph}\@seccntformat{paragraph}}\,}
  \let\old@par=\par
  \def\new@par{\let\par=\old@par\paragraph{}\let\par=\new@par}
  \let\par=\new@par
  \par
}{
  \addtocounter{secnumdepth}{-1}
}
\newenvironment{numberedsubpars}{%
  \addtocounter{secnumdepth}{1}
  \renewcommand\thesubparagraph{\arabic{section}.\arabic{paragraph}.\arabic{subparagraph}.}
  \renewcommand\@seccntformat[1]
  {\expandafter\ifx\csname##1\endcsname\subparagraph\csname 
  the##1\endcsname\else\csname the##1\endcsname\fi}
  \def\subparagraph{\par\addvspace{3.25ex \@plus1ex \@minus .2ex}{\raggedsection\normalfont\sectfont\nobreak\size@paragraph\refstepcounter{subparagraph}\@seccntformat{subparagraph}}\,}
  \let\old@@par=\par
  \def\new@@par{\let\par=\old@par\subparagraph{}\let\par=\new@@par}
  \let\par=\new@@par
  \par
}{
  \addtocounter{secnumdepth}{-1}
  \let\par=\old@@par
}

\def\emph{\textbf}

\renewcommand\thesection{\arabic{section}.}
\let\@@@section=\section
\renewcommand\section[1]{\@@@section{\MakeUppercase{#1}}}
% end of magic

\makeatother


\sloppy
\begin{document}

\setkomafont{title}{\bfseries}
\date{\large г. Минск\\2012}


\titlehead{\raggedleft \begin{minipage}{18em}%
ПРИНЯТ\\%
Учредительной конференцией\\%
Республиканского\\%
общественного объединения\\%
«Белорусское общество\\%
открытых и свободных технологий»\\%
00.00.2012%%
\end{minipage}}

\title{\Large%
УСТАВ\\%
\textbf{\large Республиканского общественного объединения}\\%
«Белорусское общество открытых и свободных технологий»\\[1ex]%
(РОО «БООСТ»)\\[4ex]%
СТАТУТ\\%
\textbf{\large Рэспубліканскага грамадскага аб'яднання}\\%
«Беларускае таварыства адкрытых і вольных тэхналогіяў»\\[1ex]%
(РГА «БТАВТ»)%
}
\maketitle

\section{Общие положения}

\begin{numberedpars}%
РОО «Белорусское общество открытых и свободных технологий» является добровольным объединением граждан, которое они образовали на
основе общности интересов для совместной реализации гражданских, социальных, культурных и иных прав. РОО «Белорусское общество открытых и свободных технологий» действует на основе принципов законности, добровольности, самостоятельности и
гласности.

Республиканское общественное объединение «Белорусское общество открытых и свободных технологий» (далее по тексту "--- РОО «Белорусское общество открытых и свободных технологий») создано и осуществляет свою деятельность в соответствии с Конституцией Республики Беларусь,
Законом Республики Беларусь «Об общественных объединениях», Указами и Декретами Президента Республики Беларусь, иными
актами законодательства Республики Беларусь и настоящим Уставом.

Наименование:
\begin{numberedsubpars}
Полное на русском языке: \emph{Общественное объединение «Белорусское общество открытых и свободных технологий»};

Полное на белорусском языке: \emph{Рэспубліканскае грамадскае аб’яднанне «Беларускае таварыства адкрытых і вольных тэхналогіяў»};

Сокращённое на русском языке: \emph{РОО «БООСТ»};

Сокращённое на белорусском языке: \emph{РГА «БТАВТ»}.
\end{numberedsubpars}

РОО «Белорусское общество открытых и свободных технологий» имеет статус республиканского общественного объединения. РОО
«Белорусское общество открытых и свободных технологий» распространяет свою деятельность на территорию Республики Беларусь.

РОО «Белорусское общество открытых и свободных технологий» является юридическим лицом, имеет обособленное имущество,
самостоятельный баланс, расчётный и другие счета в учреждениях банков, от своего имени выступает во взаимоотношениях с
юридическими и физическими лицами, может быть истцом и ответчиком в судах, имеет печать, бланки со своим наименованием,
может иметь собственную символику, зарегистрированную в установленном законодательством порядке, может иметь продукцию деятельности.

РОО «Белорусское общество открытых и свободных технологий» может участвовать в создании международных, республиканских, местных
союзов (ассоциаций) общественных объединений. РОО «Белорусское общество открытых и свободных технологий» может вступать в
международные, республиканские, местные союзы (ассоциации) общественных объединений. РОО «Белорусское общество открытых и свободных технологий» может участвовать в создании на территории иностранных государств международных общественных
объединений, союзов (ассоциаций) общественных объединений, вступать в международные общественные объединения, союзы
(ассоциации) общественных объединений, созданных на территории иностранных государств. РОО «Белорусское общество открытых и свободных технологий» может поддерживать прямые международные контакты и связи, заключать соответствующие соглашения
и предпринимать иные шаги, не противоречащие законодательству Республики Беларусь, в том числе международным договорам
Республики Беларусь.

РОО «Белорусское общество открытых и свободных технологий» ведёт делопроизводство в соответствии с законодательством, и
документы, определённые законодательством, в установленном порядке сдаются в соответствующее учреждение Национального
архивного фонда по месту нахождения юридического адреса РОО «Белорусское общество открытых и свободных технологий».

Юридический адрес РОО «Белорусское общество открытых и свободных технологий»: 220053 г. Минск, ул. Мирная 37Г, пом. 2.
\end{numberedpars}

\newpage\section{Цели, задачи, предмет и методы деятельности РОО «Белорусское общество открытых и свободных технологий»}

\begin{numberedpars}
Основной целью РОО «Белорусское общество открытых и свободных технологий» является развитие современных информационных
технологий, науки, техники и электроники, повышение уровня знаний и навыков населения в данной области.

Задачами РОО «Белорусское общество открытых и свободных технологий» являются:
\begin{numberedsubpars}
распространение культуры и принципов свободного программного и аппаратного обеспечения.

поддержка и популяризация технического творчества;

cоздание и обеспечение функционирования центров технического творчества;

создание благоприятной среды для обмена знаниями и опытом, общения и совместной работы для участников проектов,
связанных с техническим творчеством;
\end{numberedsubpars}

Предмет деятельности РОО «Белорусское общество открытых и свободных технологий»: просветительская, образовательная деятельность,
стимуляция технического творчества;

Методами деятельности РОО «Белорусское общество открытых и свободных технологий» являются:
\begin{numberedsubpars}
свободное распространение информации о деятельности сообщества;

проведение конкурсов, соревнований, обучающих семинаров, лекций, конференций, выставок, круглых столов и иных мероприятий, представляющих интерес для членов сообщества;

участие в организации выставок, конференций, круглых столов и иных аналогичных мероприятий, представляющих интерес для
членов сообщества;

создание и обеспечение работы центров технического творчества, клубов радиолюбителей-конструкторов, программистов, и
подобных;

содействие и поддержка в реализации оригинальных проектов участников сообщества;

в установленном законодательством порядке вступление в международные общественные (неправительственные и правительственные) объединения и
союзы, установление и поддержка прямых международных контактов и связей, заключение для этих целей соответствующих
соглашений;

получение финансовой и иной поддержки из различных не запрещённых законодательством Республики Беларусь источников;

иные методы, не противоречащие законодательству Республики Беларусь.
\end{numberedsubpars}
\end{numberedpars}

\section{Члены РОО «Белорусское общество открытых и свободных технологий», их права и обязанности}

\begin{numberedpars}
Членами РОО «Белорусское общество открытых и свободных технологий» могут быть граждане Республики Беларусь, иностранные граждане
и лица без гражданства (далее "--- граждане), содействующие выполнению задач РОО «Белорусское общество открытых и свободных технологий» и уплачивающие членские взносы. Граждане, не достигшие 16-летнего возраста, принимаются в члены РОО «Белорусское общество открытых и свободных технологий» с письменного разрешения родителей (опекунов, попечителей).

РОО «Белорусское общество открытых и свободных технологий» имеет фиксированное членство. Учёт членов РОО «Белорусское общество открытых и свободных технологий» ведёт Правление РОО «Белорусское общество открытых и свободных технологий».

Приём в члены РОО «Белорусское общество открытых и свободных технологий» осуществляется на основании заявления, поданного
гражданином, и после уплаты им вступительного взноса. Заявитель становится членом РОО «Белорусское общество открытых и свободных технологий» после принятия заявления от него ответственным лицом и уплаты вступительного взноса.

Размер вступительного и членских взносов, порядок их уплаты определяется Правлением. Правление может принять решение об
освобождении отдельных граждан от уплаты вступительного и членских взносов, для отдельных членов РОО «Белорусское общество открытых и свободных технологий» размер вступительного и членских взносов может быть снижен.

Учёт членов РОО «Белорусское общество открытых и свободных технологий» осуществляется лицом, назначенным специально для этих
целей Правлением, путём ведения базы данных членов, которая редактируется по мере необходимости и обновляется при
вступлении и выбытии членов.

Члены РОО «Белорусское общество открытых и свободных технологий» имеют право:
\begin{numberedsubpars}
быть избранным в выборные органы РОО «Белорусское общество открытых и свободных технологий» по достижению ими 18-летнего
возраста;

участвовать в конкурсах, выставках и других мероприятиях, проводимых РОО «Белорусское общество открытых и свободных технологий»;

участвовать в работе создаваемых РОО «Белорусское общество открытых и свободных технологий» клубов, центров технического
творчества на условиях, определяемых правилами клуба, центра;

получать информацию о деятельности руководящих органов, контрольно-ревизионных, иных органов РОО «Белорусское общество открытых и свободных технологий», вносить в данные органы предложения по улучшению работы РОО «Белорусское общество открытых и свободных технологий» и участвовать в их реализации;

вносить предложения в повестку дня Конференции РОО «Белорусское общество открытых и свободных технологий»;
участвовать в обсуждении вопросов, касающихся их прав и обязанностей, обращаться с заявлениями в соответствующие органы
РОО «Белорусское общество открытых и свободных технологий»;

быть членом общественных объединений, чья деятельность не противоречит настоящему Уставу, с соблюдением требований
настоящего Устава и законодательства;

свободно выходить из членов РОО «Белорусское общество открытых и свободных технологий»;

иметь другие права, предоставляемые настоящим Уставом и законодательством.
\end{numberedsubpars}

Все члены РОО «Белорусское общество открытых и свободных технологий» при голосовании имеют один голос.

Члены РОО «Белорусское общество открытых и свободных технологий» обязаны:
\begin{numberedsubpars}
выполнять требования настоящего Устава;

не совершать действий, подрывающих авторитет или наносящих материальный вред РОО «Белорусское общество открытых и свободных технологий»;

своевременно уплачивать членские взносы;

исполнять иные обязанности, предусмотренные настоящим Уставом и законодательством.
\end{numberedsubpars}

За активное участие в работе РОО «Белорусское общество открытых и свободных технологий» его члены могут быть поощрены по решению
Правления РОО «Белорусское общество открытых и свободных технологий»: объявлением благодарности, награждением грамотами, знаками
РОО «Белорусское общество открытых и свободных технологий», памятными подарками, призами в виде денежной премии, иным образом в
соответствии с настоящим уставом и законодательством.

Прекращение членства в РОО «Белорусское общество открытых и свободных технологий» происходит в случаях:
\begin{numberedsubpars}
выхода из членов РОО «Белорусское общество открытых и свободных технологий»;

исключения из членов РОО «Белорусское общество открытых и свободных технологий»;

в иных случаях, предусмотренных законодательством.
\end{numberedsubpars}

Выход из членов РОО «Белорусское общество открытых и свободных технологий» осуществляется по заявлению гражданина "--- члена РОО
«Белорусское общество открытых и свободных технологий».

За нарушение требований настоящего Устава, повлёкшее подрыв авторитета РОО «Белорусское общество открытых и свободных технологий»
или причинение ему материального вреда, а также неуплату членских взносов в срок, определённый порядком их уплаты,
граждане могут быть исключены из членов РОО «Белорусское общество открытых и свободных технологий». Граждане могут быть исключены
из членов РОО «Белорусское общество открытых и свободных технологий» за совершение проступка, дискредитирующего звание члена РОО
«Белорусское общество открытых и свободных технологий».

Вопросы об исключении из числа членов РОО «Белорусское общество открытых и свободных технологий» решает Правление РОО «Белорусское общество открытых и свободных технологий».

Лица, исключённые из членов РОО «Белорусское общество открытых и свободных технологий», имеют право в течение 15 (пятнадцати)
рабочих дней подать жалобу Ревизору РОО «Белорусское общество открытых и свободных технологий», который обязан рассмотреть её в
срок, установленный законодательством. До принятия решения по жалобе лицо, подавшее её, считается членом РОО «Белорусское общество открытых и свободных технологий».

Члены РОО «Белорусское общество открытых и свободных технологий» не отвечают по обязательствам РОО «Белорусское общество открытых и свободных технологий», членами которого они являются.
\end{numberedpars}

\section{Высший и выборные органы РОО «Белорусское общество открытых и свободных технологий», их компетенция}

\begin{numberedpars}
РОО «Белорусское общество открытых и свободных технологий» является цельным общественным объединением граждан, не имеющим в своём
составе организационных структур.

Структуру РОО «Белорусское общество открытых и свободных технологий» составляют:
\begin{numberedsubpars}
высший орган РОО «Белорусское общество открытых и свободных технологий» "--- Конференция РОО «Белорусское общество открытых и свободных технологий» (далее "--- Конференция);

руководящий орган "--- Правление РОО «Белорусское общество открытых и свободных технологий» (далее "--- Правление), из числа членов
которого избирается Председатель Правления (далее "--- Председатель);

контрольно-ревизионный орган "--- Ревизор РОО «Белорусское общество открытых и свободных технологий» (далее "--- Ревизор).
\end{numberedsubpars}

Заседания Конференции проводятся по мере необходимости, но не реже, чем один раз в календарный год.

Внеочередные заседания Конференции созываются:
\begin{numberedsubpars}
по решению Правления РОО «Белорусское общество открытых и свободных технологий»;

по требованию Ревизора РОО «Белорусское общество открытых и свободных технологий»;

по требованию 50\% и более членов РОО «Белорусское общество открытых и свободных технологий».
\end{numberedsubpars}

К исключительной компетенции Конференции относится:
\begin{numberedsubpars}
определение основных направлений деятельности РОО «Белорусское общество открытых и свободных технологий»;

утверждение названия РОО «Белорусское общество открытых и свободных технологий»;

утверждение Устава РОО «Белорусское общество открытых и свободных технологий», внесение любых изменений и дополнений в Устав РОО
«Белорусское общество открытых и свободных технологий»;

ежегодное избрание членов Правления, избрание Ревизора РОО «Белорусское общество открытых и свободных технологий» и досрочное
прекращение их полномочий;

принятие решения о реорганизации или ликвидации РОО «Белорусское общество открытых и свободных технологий»;

решение иных вопросов, предусмотренных законодательством и настоящим Уставом.
\end{numberedsubpars}

Вопросы, отнесённые к исключительной компетенции Конференции, не могут быть переданы на решение Правлению либо
Председателю Правления РОО «Белорусское общество открытых и свободных технологий».

Конференция признается правомочной (имеет кворум), если на ней присутствуют более чем две трети делегатов РОО «Белорусское общество открытых и свободных технологий». В случае отсутствия установленного кворума ежегодная Конференция должна, а
внеочередная "--- может быть проведена повторно с той же повесткой дня. Повторная Конференция имеет кворум, если на ней
присутствуют более половины делегатов РОО «Белорусское общество открытых и свободных технологий».

Решения Конференции принимаются простым большинством голосов делегатов, принявших участие в Конференции. При равенстве
голосов решающим является голос Председателя Конференции, который голосует последним.

Конференция решает вопросы на своих заседаниях. Подготовку, созыв и проведение Конференции осуществляет Правление РОО
«Белорусское общество открытых и свободных технологий».

Решение о созыве и проведении очередной Конференции должно быть принято Правлением не позднее 30 дней после окончания
отчётного периода.

Внеочередная Конференция должна быть проведена не позднее 30 дней с момента принятия Правлением решения о созыве и
проведении этой Конференции. Правление не позднее пятнадцати дней с момента получения требования о проведении
внеочередной Конференции обязано рассмотреть данное требование и принять решение о созыве и проведении внеочередной
Конференции либо мотивированное решение об отказе в её созыве и проведении. Решение Правления о созыве и проведении
внеочередной Конференции либо мотивированное решение об отказе в её созыве и проведении направляются лицам, требующим её
созыва, посредством почтовой связи или иным способом, обеспечивающим подлинность передаваемых и принимаемых сообщений и
их документальное подтверждение, не позднее пяти дней с момента принятия такого решения.

\shortpage{}Правление в двухмесячный срок принимает решение о проведении Конференции, в котором должны быть определены:
дата, время и место (с указанием адреса) проведения Конференции;
повестка дня Конференции;

Решение о проведении Конференции может содержать и иные сведения, указание которых целесообразно в каждом конкретном
случае.

Лица, имеющие право на участие в Конференции, извещаются о принятом решении о проведении Конференции Правлением не менее
чем за десять дней до даты её проведения. Каждый член РОО «Белорусское общество открытых и свободных технологий» извещается по
адресу, указанному в списке членов РОО «Белорусское общество открытых и свободных технологий» посредством почтовой связи или иным
способом, обеспечивающим подлинность передаваемых и принимаемых сообщений и их документальное подтверждение.

Извещение о проведении повторной Конференции должно быть направлено не менее чем за десять дней до даты её проведения.

Делегатами, принявшими участие в Конференции, считаются зарегистрировавшиеся для участия в ней. Регистрация делегатов,
имеющих право на участие в Конференции, осуществляется при предъявлении ими документов, удостоверяющих личность, и при
наличии делегата в списке делегатов Конференции РОО «Белорусское общество открытых и свободных технологий», который ведёт
Правление РОО «Белорусское общество открытых и свободных технологий». После регистрации делегатов определяется правомочность
(наличие кворума) этой Конференции. Делегаты, не прошедшие регистрацию, не вправе принимать участие в голосовании на
Конференции РОО «Белорусское общество открытых и свободных технологий».

Предложение в повестку дня Конференции должно содержать имя физического лица и формулировку каждого из предлагаемых в
повестку дня вопросов, а также его подпись. Предложение в повестку дня о выдвижении кандидатов в избираемые (образуемые)
органы РОО «Белорусское общество открытых и свободных технологий» должно также содержать имя каждого предлагаемого кандидата,
наименование органа РОО «Белорусское общество открытых и свободных технологий», для избрания в который он предлагается.

Повестка дня Конференции формируется Председателем по своему усмотрению, а также на основании предложений членов РОО
«Белорусское общество открытых и свободных технологий». 

Председатель не позднее десяти дней после окончания срока, установленного для поступления предложений в повестку дня,
обязан рассмотреть эти предложения и принять решение об их учете либо об отказе в их принятии в случаях, предусмотренных
законодательством. В случае отказа в принятии предложений Председатель должен направить лицу, внесшему эти предложения,
своё мотивированное решение не позднее пяти дней с даты принятия такого решения.

Конференция проводится в очной форме, которая предусматривает совместное присутствие лиц, имеющих право на участие в
этом заседании, при обсуждении вопросов повестки дня Конференции и принятии решений по ним.

Форма голосования при приеме решений Конференции определяется самой Конференцией.

Решения, принятые Конференцией, оглашаются на этой Конференции либо доводятся до сведения членов РОО «Белорусское общество открытых и свободных технологий» не позднее десяти дней после даты подписания протокола этой Конференции посредством
направления в адрес членов РОО «Белорусское общество открытых и свободных технологий» копии указанного протокола. Члены РОО
«Белорусское общество открытых и свободных технологий» имеют право ознакомиться с протоколом Конференции по юридическому адресу
РОО «Белорусское общество открытых и свободных технологий».

Правление является руководящим выборным органом РОО «Белорусское общество открытых и свободных технологий». Правление
осуществляет руководство деятельностью РОО «Белорусское общество открытых и свободных технологий» в период между заседаниями
Конференции.

\shortpage{}\shortpage{}%
Правление состоит из пяти членов РОО «Белорусское общество открытых и свободных технологий», которые избираются ежегодно путем
открытого голосования делегатов Конференции РОО «Белорусское общество открытых и свободных технологий». В состав Правления могут
быть избраны только члены РОО «Белорусское общество открытых и свободных технологий», достигшие восемнадцатилетнего возраста. Не
допускается одновременное занятие членом РОО «Белорусское общество открытых и свободных технологий» должности Ревизора и члена
Правления РОО «Белорусское общество открытых и свободных технологий».

%\pagebreak{}%
Правление:
\begin{numberedsubpars}
принимает решение о приобретении членства в РОО «Белорусское общество открытых и свободных технологий» и об исключении из членов
РОО «Белорусское общество открытых и свободных технологий»;

в период между заседаниями Конференции вносит в Устав РОО «Белорусское общество открытых и свободных технологий» изменения и
(или) дополнения, связанные с изменением его юридического адреса либо обусловленные изменениями в законодательстве
Республики Беларусь;

ведет список членов РОО «Белорусское общество открытых и свободных технологий»;

принимает решение о создании других юридических лиц, в том числе коммерческих организаций, а также решение об участии РОО
«Белорусское общество открытых и свободных технологий» в других юридических лицах;

утверждает регламент проведения Конференции;

определяет норму делегирования и порядок избрания делегатов для участия в Конференции;

принимает иные решения, обязательные для всех органов и членов РОО «Белорусское общество открытых и свободных технологий», не
отнесенные к исключительной компетенции Конференции.
\end{numberedsubpars}

Правление проводит заседания по мере необходимости, но не реже трёх раз в год. Извещение о проведении заседания
Правления должно быть направлено Председателем Правления всем членам Правления не менее чем за десять дней до даты его
проведения.

Председатель Правления формирует вопросы повестки дня на основании предложений членов Правления, направивших свои
предложения о рассмотрении отдельных вопросов деятельности РОО «Белорусское общество открытых и свободных технологий».

Заседание Правления считается правомочным, если на нем присутствует не менее половины членов Правления. Решение
Правления считается принятым, если за него проголосовало более половины присутствующих членов. При равном количестве
голосов голос Председателя Правления имеет решающее значение.

Заседание Правления ведет Председатель Правления. Ведение протокола Правления обеспечивает Секретарь, избираемый на
заседании Правления. Протокол составляется в двух экземплярах и подписывается Председателем и Секретарем. 

Решения Правления могут приниматься открытым голосованием либо тайным голосованием бюллетенями.

Решения, принятые Правлением, оформляются протоколом, оглашаются на заседании Правления и доводятся до сведения членов
РОО «Белорусское общество открытых и свободных технологий» не позднее десяти рабочих дней после даты подписания протокола
посредством направления в адрес членов РОО «Белорусское общество открытых и свободных технологий» копии указанного протокола.
Члены РОО «Белорусское общество открытых и свободных технологий» имеют право ознакомиться с протоколом Правления по юридическому
адресу РОО «Белорусское общество открытых и свободных технологий».

Внеочередные заседания Правления созываются по решению Председателя или по инициативе большинства его членов.
Решения Правления могут быть обжалованы любым членом РОО «Белорусское общество открытых и свободных технологий» Ревизору в
течение 15 (пятнадцати) рабочих дней со дня уведомления члена РОО «Белорусское общество открытых и свободных технологий» о
принятом на заседании Правления решении. Ревизор обязан рассмотреть указанную жалобу в срок, установленный
законодательством. 

\shortpage%
Председатель Правления:
\begin{numberedsubpars}
осуществляет текущее руководство деятельностью РОО «Белорусское общество открытых и свободных технологий»;

в пределах своей компетенции без доверенности действует от имени РОО «Белорусское общество открытых и свободных технологий», в
том числе представляет интересы РОО «Белорусское общество открытых и свободных технологий» и подписывает необходимые документы от
имени РОО «Белорусское общество открытых и свободных технологий»;

нанимает и увольняет штатных работников в соответствии с законодательством и условиями контрактов (договоров), в том
числе гражданско-правовых; утверждает штатное расписание РОО «Белорусское общество открытых и свободных технологий»;

представляет РОО «Белорусское общество открытых и свободных технологий» без доверенности в отношениях с государственными органами
Республики Беларусь и других государств, юридическими и физическими лицами;

в пределах своих полномочий, определённых в договоре (контракте), приобретает от имени РОО «Белорусское общество открытых и свободных технологий» и распоряжается имуществом, в том числе средствами РОО «Белорусское общество открытых и свободных технологий»;

открывает в банках текущий (расчётный), валютный и иные счета предусмотренные законодательством, закрывает счета РОО
«Белорусское общество открытых и свободных технологий», распоряжается денежными средствами РОО «Белорусское общество открытых и свободных технологий» на счетах в пределах, установленных договором (контрактом), настоящим Уставом и законодательством;

выдаёт доверенности;

организует подготовку, созыв и проведение заседания Правления РОО «Белорусское общество открытых и свободных технологий»;

председательствует на заседаниях Правления РОО «Белорусское общество открытых и свободных технологий»;

заключает от имени РОО «Белорусское общество открытых и свободных технологий» договоры и обеспечивает их выполнение;

принимает иные решения, обязательные для всех органов и членов РОО «Белорусское общество открытых и свободных технологий», не
отнесенные к исключительной компетенции Конференции и Правления РОО «Белорусское общество открытых и свободных технологий».
\end{numberedsubpars}

Председатель в пределах своих полномочий издаёт приказы и даёт указания, обязательные для всех штатных работников РОО
«Белорусское общество открытых и свободных технологий».

Решения Председателя Правления могут быть обжалованы любым членом РОО «Белорусское общество открытых и свободных технологий»
Ревизору в течение 15 (пятнадцати) рабочих дней со дня, когда член РОО «Белорусское общество открытых и свободных технологий»
узнал о принятом Председателем решении. Ревизор обязан рассмотреть указанную жалобу в срок, установленный
законодательством.

Ревизор РОО «Белорусское общество открытых и свободных технологий» осуществляет внутреннюю проверку финансово-хозяйственной
деятельности РОО «Белорусское общество открытых и свободных технологий», а также внутренний контроль за соответствием
деятельности РОО «Белорусское общество открытых и свободных технологий» законодательству и его Уставу.

Ревизор подотчётен Конференции.

Ревизор РОО «Белорусское общество открытых и свободных технологий»:
\begin{numberedsubpars}
контролирует и осуществляет ревизии деятельности выборных органов и должностных лиц РОО «Белорусское общество открытых и свободных технологий»;

контролирует учет материальных ценностей и денежных средств РОО «Белорусское общество открытых и свободных технологий»;

проверяет организацию делопроизводства РОО «Белорусское общество открытых и свободных технологий», сроки, законность и
обоснованность ответов на заявления, жалобы и письма членов РОО «Белорусское общество открытых и свободных технологий»;

по мере необходимости, но не реже одного раза в год проводит ревизию деятельности РОО «Белорусское общество открытых и свободных технологий» и оформляет результат проверки заключением;

проводит ревизию деятельности РОО «Белорусское общество открытых и свободных технологий» по решению Конференции в установленные
им сроки;

по своей инициативе проводит ревизию деятельности РОО «Белорусское общество открытых и свободных технологий».
\end{numberedsubpars}

Продолжительность ревизии или проверки не должна превышать 30 (тридцати) дней.

%\pagebreak{}
Ревизор РОО «Белорусское общество открытых и свободных технологий» в случае выявления нарушений обязан:
\begin{numberedsubpars}
представить заключение ревизии или проверки либо отдельные их выводы и предложения Правлению, которое в соответствии с
его компетенцией в двухнедельный срок обязано принять меры по устранению допущенных нарушений;

потребовать созыва внеочередной Конференции, если по выявленным в ходе ревизии или проверки фактам нарушений решение
может быть принято только этой Конференцией.
\end{numberedsubpars}

Ревизор может присутствовать на заседаниях Правления с правом совещательного голоса.

Результаты проверок Ревизора оформляются актами и (или) справками.

Компетенция Ревизора по вопросам, не предусмотренным настоящим Уставом, определяются решением Конференции.
\end{numberedpars}

\section{Имущество и средства РОО «Белорусское общество открытых и свободных технологий»}

\begin{numberedpars}
РОО «Белорусское общество открытых и свободных технологий» может иметь в собственности любое имущество, необходимое для
материального обеспечения его деятельности, предусмотренной настоящим Уставом, за исключением объектов, которые согласно
законодательству могут находиться только в собственности государства.

Собственником имущества РОО «Белорусское общество открытых и свободных технологий» является РОО «Белорусское общество открытых и свободных технологий». Члены РОО «Белорусское общество открытых и свободных технологий» не имеют права собственности на долю в его
имуществе. Имущество, переданное в собственность РОО «Белорусское общество открытых и свободных технологий» его членами, другими
лицами, является собственностью данного объединения.

%\pagebreak{}
Денежные средства РОО «Белорусское общество открытых и свободных технологий» формируются из:
\begin{numberedsubpars}
вступительных и членских взносов;

поступлений от проводимых в уставных целях лекций, выставок, семинаров, конференций, спортивных и других мероприятий;

доходов от предпринимательской деятельности, осуществляемой в установленном настоящим Уставом и законодательством
порядке;

добровольных пожертвований;

иных источников, не запрещённых законодательством.
\end{numberedsubpars}

Денежные средства РОО «Белорусское общество открытых и свободных технологий» расходуются на:
\begin{numberedsubpars}
выполнение задач, стоящих перед РОО «Белорусское общество открытых и свободных технологий»;

финансирование планов, программ, проектов исследований в соответствии с уставными целями и задачами РОО «Белорусское общество открытых и свободных технологий»;

обеспечение работы выборных органов РОО «Белорусское общество открытых и свободных технологий»;

выплату заработной платы, улучшение социально-бытовых условий штатных сотрудников исполнительного аппарата и членов РОО
«Белорусское общество открытых и свободных технологий», их премирование и поощрение в соответствии с настоящим Уставом и
законодательством;

создание юридических лиц, деятельность которых направлена на решение уставных задач;

развитие материально-технической базы РОО «Белорусское общество открытых и свободных технологий»;

благотворительную деятельность;

на иные цели, предусмотренные настоящим Уставом, законодательством.
\end{numberedsubpars}

Денежные средства и иное имущество РОО «Белорусское общество открытых и свободных технологий» не могут перераспределяться между
членами этого РОО «Белорусское общество открытых и свободных технологий» и используются только для выполнения уставных целей и
задач.

РОО «Белорусское общество открытых и свободных технологий» не отвечает по обязательствам своих членов. Члены РОО «Белорусское общество открытых и свободных технологий» не отвечают по обязательствам РОО «Белорусское общество открытых и свободных технологий»,
членами которого они являются.

Права владения, пользования, распоряжения имуществом, находящимся в собственности РОО «Белорусское общество открытых и свободных технологий», осуществляет Председатель Правления РОО «Белорусское общество открытых и свободных технологий» в соответствии с
законодательством и настоящим Уставом в пределах, установленных Правлением. 
\end{numberedpars}

\shortpage\section{Права РОО «Белорусское общество открытых и свободных технологий»}

\begin{numberedpars}
РОО «Белорусское общество открытых и свободных технологий» не отвечает по обязательствам своих членов. 

РОО «Белорусское общество открытых и свободных технологий» имеет право:
\begin{numberedsubpars}
осуществлять деятельность, направленную на достижение уставных целей;

беспрепятственно получать и распространять информацию, имеющую отношение к деятельности РОО «Белорусское общество открытых и свободных технологий»;

пользоваться государственными средствами массовой информации в порядке, установленном законодательством;

учреждать собственные средства массовой информации и осуществлять издательскую деятельность в порядке, установленном
законодательством;

защищать права и законные интересы, а также представлять законные интересы своих членов в государственных органах и иных
организациях;

поддерживать связи с другими общественными объединениями, союзами и ассоциациями;

создавать союзы;

осуществлять в установленном порядке предпринимательскую деятельность лишь постольку, поскольку она необходима для
уставных целей, ради которых оно создано, соответствует этим целям и отвечает предмету деятельности. Указанная
деятельность может осуществляться РОО «Белорусское общество открытых и свободных технологий» только посредством образования
коммерческих организаций и (или) участия в них;

осуществлять иные права в соответствии с настоящим Уставом и действующим законодательством.
\end{numberedsubpars}
\end{numberedpars}

\section{Реорганизация и ликвидация РОО «Белорусское общество открытых и свободных технологий»}

\begin{numberedpars}
Прекращение деятельности РОО «Белорусское общество открытых и свободных технологий» происходит в случае его реорганизации либо
ликвидации.

Реорганизация или ликвидация РОО «Белорусское общество открытых и свободных технологий» осуществляется по решению Конференции РОО
«Белорусское общество открытых и свободных технологий», принятому не менее 2/3 голосов всех членов РОО «Белорусское общество открытых и свободных технологий» либо их делегатов, в соответствии с законодательством Республики Беларусь.

Ликвидация РОО «Белорусское общество открытых и свободных технологий» также осуществляется по решению суда в случаях и порядке,
предусмотренных законодательством Республики Беларусь.

Ликвидационная комиссия создаётся органом, принявшим решение о ликвидации.

В случае ликвидации РОО «Белорусское общество открытых и свободных технологий», принадлежащее ему имущество и денежные средства,
оставшиеся после удовлетворения требований кредиторов, используются по решению ликвидационной комиссии на цели,
предусмотренные настоящим уставом.
\end{numberedpars}

\end{document}
