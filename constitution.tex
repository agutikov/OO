\documentclass[a4paper,fontsize=14pt,titlepage]{scrartcl}

\usepackage[T2A]{fontenc}
\usepackage[utf8]{inputenc}
\usepackage[russian]{babel}
\usepackage{indentfirst}
\usepackage{graphicx}
\usepackage[left=2cm,top=2cm,right=1.7cm,bottom=2cm,nohead,nofoot]{geometry}
\usepackage[font={small}]{caption}
\usepackage{wrapfig}

\usepackage{enumitem}
\setlist{nolistsep}
%\setlist[1]{\labelindent=\parindent}
%\setlist[itemize]{leftmargin=*}
\setlist[itemize,1]{label=---}

\setkomafont{section}{\sffamily \large}
\setkomafont{paragraph}{\normalfont \bfseries}
\setkomafont{subparagraph}{\normalfont \bfseries}
\setkomafont{title}{\rmfamily \huge}

\newcommand{\longpage}{\enlargethispage{\baselineskip}}
\newcommand{\shortpage}{\enlargethispage{-\baselineskip}}

% deep magic follows, don't touch it!
\makeatletter
\@newctr{paragraph}[section]
\@newctr{subparagraph}[paragraph]
\newenvironment{numberedpars}{%
  \addtocounter{secnumdepth}{1}
  \renewcommand\theparagraph{\arabic{section}.\arabic{paragraph}.}
  \renewcommand\@seccntformat[1]
  {\expandafter\ifx\csname##1\endcsname\paragraph\csname 
  the##1\endcsname\else\csname the##1\endcsname\quad\fi}
  \let\old@par=\par
  \def\new@par{\let\par=\old@par\paragraph{}\let\par=\new@par}
  \let\par=\new@par
  \par
}{
  \addtocounter{secnumdepth}{-1}
}
\newenvironment{numberedsubpars}{%
  \addtocounter{secnumdepth}{1}
  \renewcommand\thesubparagraph{\arabic{section}.\arabic{paragraph}.\arabic{subparagraph}.}
  \renewcommand\@seccntformat[1]
  {\expandafter\ifx\csname##1\endcsname\subparagraph\csname 
  the##1\endcsname\else\csname the##1\endcsname\quad\fi}
  \let\old@@par=\par
  \def\new@@par{\let\par=\old@par\subparagraph{}\let\par=\new@@par}
  \let\par=\new@@par
  \par
}{
  \addtocounter{secnumdepth}{-1}
  \let\par=\old@@par
}

\def\emph{\textbf}

\let\@@@section=\section
\renewcommand\section[1]{\@@@section{\MakeUppercase{#1}}}
% end of magic

\makeatother


\sloppy
\begin{document}

\setkomafont{title}{\bfseries}

\date{\large г. Минск\\2012}

\titlehead{\raggedleft \begin{minipage}{15em}%
ПРИНЯТ\\%
Учредительной конференцией\\%
Общественного объединения\\%
«Открытая лаборатория\\%
технического творчества»\\%
00.00.2012%%
\end{minipage}}

\title{\Large%
УСТАВ\\%
\textbf{\large Общественного объединения}\\%
«Открытая лаборатория технического творчества»\\[1ex]%
(ОО «Открытая лаборатория\\%
технического творчества»)\\[4ex]%
СТАТУТ\\%
\textbf{\large Грамадскага аб'яднання}\\%
«Адкрытая лабараторыя тэхнічнай творчасці»\\[1ex]%
(ГА «Адкрытая лабараторыя\\%
тэхнічнай творчасці»)%
}
\maketitle

\section{Общие положения}

\begin{numberedpars}%
ОО «Открытая лаборатория технического творчества» является добровольным объединением граждан, которое они образовали на
основе общности интересов для совместной реализации гражданских, социальных, культурных и иных прав. ОО «Открытая
лаборатория технического творчества» действует на основе принципов законности, добровольности, самостоятельности и
гласности.

Общественное объединение «Открытая лаборатория технического творчества» (далее по тексту "--- ОО «Открытая лаборатория
технического творчества») создано и осуществляет свою деятельность в соответствии с Конституцией Республики Беларусь,
Законом Республики Беларусь «Об общественных объединениях», Указами и Декретами Президента Республики Беларусь, иными
актами законодательства Республики Беларусь и настоящим Уставом.

Наименование:
\begin{numberedsubpars}
Полное на русском языке: \emph{Общественное объединение «Открытая лаборатория технического творчества»};

Полное на белорусском языке: \emph{Грамадскае аб’яднанне «Адкрытая лабараторыя тэхнічнай творчасці»};

Сокращенное на русском языке: \emph{ОО «Открытая лаборатория технического творчества»};

Сокращенное на белорусском языке: \emph{ГА «Адкрытая лабараторыя тэхнічнай творчасці»}.
\end{numberedsubpars}

ОО «Открытая лаборатория технического творчества» имеет статус местного, регионального общественного объединения. ОО
«Открытая лаборатория технического творчества» распространяет свою деятельность на территорию г. Минска и Минской
области.

ОО «Открытая лаборатория технического творчества» является юридическим лицом, имеет обособленное имущество,
самостоятельный баланс, расчетный и другие счета в учреждениях банков, от своего имени выступает во взаимоотношениях с
юридическими и физическими лицами, может быть истцом и ответчиком в судах, имеет печать, бланки со своим наименованием,
может иметь собственную символику, зарегистрированную в установленном законодательством порядке.

 ОО «Открытая лаборатория технического творчества» может участвовать в создании международных, республиканских, местных
союзов (ассоциаций) общественных объединений. ОО «Открытая лаборатория технического творчества» может вступать в
международные, республиканские, местные союзы (ассоциации) общественных объединений. ОО «Открытая лаборатория
технического творчества» может участвовать в создании на территории иностранных государств международных общественных
объединений, союзов (ассоциаций) общественных объединений, вступать в международные общественные объединения, союзы
(ассоциации) общественных объединений, созданных на территории иностранных государств. ОО «Открытая лаборатория
технического творчества» может поддерживать прямые международные контакты и связи, заключать соответствующие соглашения
и предпринимать иные шаги, не противоречащие законодательству Республики Беларусь, в том числе международным договорам
Республики Беларусь.

ОО «Открытая лаборатория технического творчества» ведет делопроизводство в соответствии с законодательством, и
документы, определенные законодательством, в установленном порядке сдаются в соответствующее учреждение Национального
архивного фонда по месту нахождения юридического адреса ОО «Открытая лаборатория технического творчества».
Юридический адрес ОО «Открытая лаборатория технического творчества»: 220053 г. Минск, ул. Мирная 37Г, пом. 2.

\end{numberedpars}

\newpage\section{Цели, задачи, предмет и методы деятельности ОО «Открытая лаборатория технического творчества»}

\begin{numberedpars}
Основной целью ОО «Открытая лаборатория технического творчества» является развитие современных информационных
технологий, науки, техники и электроники, повышение уровня знаний и навыков населения в данной области.

Задачами ОО «Открытая лаборатория технического творчества» являются:
\begin{numberedsubpars}
поддержка и популяризация технического творчества;

cоздание и обеспечение функционирования центров технического творчества;

создание благоприятной среды для обмена знаниями и опытом, общения и совместной работы для участников проектов,
связанных с техническим творчеством;

распространение культуры и принципов свободного программного и аппаратного обеспечения.
\end{numberedsubpars}

Предмет деятельности ОО «Открытая лаборатория технического творчества»: просветительская, образовательная деятельность,
стимуляция технического творчества;

Методами деятельности ОО «Открытая лаборатория технического творчества» являются:
\begin{numberedsubpars}
свободное распространение информации о деятельности сообщества;

проведение обучающих семинаров и лекций;

участие в организации выставок, конференций, круглых столов и иных аналогичных мероприятий, представляющих интерес для
членов технического сообщества;

проведение конкурсов и соревнований;

создание и обеспечение работы центров технического творчества, клубов радиолюбителей-конструкторов, программистов, и
подобных;

содействие и поддержка в реализации оригинальных проектов участников сообщества;

в установленном законодательством порядке вступление в международные общественные (неправительственные) объединения и
союзы, установление и поддержка прямых международных контактов и связей, заключение для этих целей соответствующих
соглашений;

получение финансовой и иной поддержки из различных не запрещенных законодательством Республики Беларусь источников;

иные методы, не противоречащие законодательству Республики Беларусь.
\end{numberedsubpars}
\end{numberedpars}

\section{Члены ОО «Открытая лаборатория технического творчества», их права и обязанности}

\begin{numberedpars}
Членами ОО «Открытая лаборатория технического творчества» могут быть граждане Республики Беларусь, иностранные граждане
и лица без гражданства (далее "--- граждане), содействующие выполнению задач ОО «Открытая лаборатория технического
творчества» и уплачивающие членские взносы. Граждане, не достигшие 16-летнего возраста, принимаются в члены ОО «Открытая
лаборатория технического творчества» с письменного разрешения родителей (опекунов, попечителей).

ОО «Открытая лаборатория технического творчества» имеет фиксированное членство. Учет членов ОО «Открытая лаборатория
технического творчества» ведет Правление ОО «Открытая лаборатория технического творчества».

Прием в члены ОО «Открытая лаборатория технического творчества» осуществляется на основании заявления, поданного
гражданином, и после уплаты им вступительного взноса. Заявитель становится членом ОО «Открытая лаборатория технического
творчества» после принятия заявления от него ответственным лицом и уплаты вступительного взноса.

Размер вступительного и членских взносов, порядок их уплаты определяется Правлением. Правление может принять решение об
освобождении отдельных граждан от уплаты вступительного и членских взносов, для отдельных членов ОО «Открытая
лаборатория технического творчества» размер вступительного и членских взносов может быть снижен.

Учет членов ОО «Открытая лаборатория технического творчества» осуществляется лицом, назначенным специально для этих
целей Правлением, путем ведения базы данных членов, которая редактируется по мере необходимости и обновляется при
вступлении и выбытии членов.

Члены ОО «Открытая лаборатория технического творчества» имеют право:
\begin{numberedsubpars}
быть избранным в выборные органы ОО «Открытая лаборатория технического творчества» по достижению ими 18-летнего
возраста;

участвовать в конкурсах, выставках и других мероприятиях, проводимых ОО «Открытая лаборатория технического творчества»;

участвовать в работе создаваемых ОО «Открытая лаборатория технического творчества» клубов, центров технического
творчества на условиях, определяемых правилами клуба, центра;

получать информацию о деятельности руководящих органов, контрольно-ревизионных, иных органов ОО «Открытая лаборатория
технического творчества», вносить в данные органы предложения по улучшению работы ОО «Открытая лаборатория технического
творчества» и участвовать в их реализации;

вносить предложения в повестку дня Конференции ОО «Открытая лаборатория технического творчества»;
участвовать в обсуждении вопросов, касающихся их прав и обязанностей, обращаться с заявлениями в соответствующие органы
ОО «Открытая лаборатория технического творчества»;

быть членом общественных объединений, чья деятельность не противоречит настоящему Уставу, с соблюдением требований
настоящего Устава и законодательства;

свободно выходить из членов ОО «Открытая лаборатория технического творчества»;

иметь другие права, предоставляемые настоящим Уставом и законодательством.
\end{numberedsubpars}

Все члены ОО «Открытая лаборатория технического творчества» при голосовании имеют один голос.

Члены ОО «Открытая лаборатория технического творчества» обязаны:
\begin{numberedsubpars}
выполнять требования настоящего Устава;

не совершать действий, подрывающих авторитет или наносящих материальный вред ОО «Открытая лаборатория технического
творчества»;

своевременно уплачивать членские взносы;

исполнять иные обязанности, предусмотренные настоящим Уставом и законодательством.
\end{numberedsubpars}

За активное участие в работе ОО «Открытая лаборатория технического творчества» его члены могут быть поощрены по решению
Правления ОО «Открытая лаборатория технического творчества»: объявлением благодарности, награждением грамотами, знаками
ОО «Открытая лаборатория технического творчества», памятными подарками, призами в виде денежной премии, иным образом в
соответствии с настоящим уставом и законодательством.

Прекращение членства в ОО «Открытая лаборатория технического творчества» происходит в случаях:
\begin{numberedsubpars}
выхода из членов ОО «Открытая лаборатория технического творчества»;

исключения из членов ОО «Открытая лаборатория технического творчества»;

в иных случаях, предусмотренных законодательством.
\end{numberedsubpars}

Выход из членов ОО «Открытая лаборатория технического творчества» осуществляется по заявлению гражданина "--- члена ОО
«Открытая лаборатория технического творчества».

За нарушение требований настоящего Устава, повлекшее подрыв авторитета ОО «Открытая лаборатория технического творчества»
или причинение ему материального вреда, а также неуплату членских взносов в срок, определённый порядком их уплаты,
граждане могут быть исключены из членов ОО «Открытая лаборатория технического творчества». Граждане могут быть исключены
из членов ОО «Открытая лаборатория технического творчества» за совершение проступка, дискредитирующего звание члена ОО
«Открытая лаборатория технического творчества».

Вопросы об исключении из числа членов ОО «Открытая лаборатория технического творчества» решает Правление ОО «Открытая
лаборатория технического творчества».

Лица, исключенные из членов ОО «Открытая лаборатория технического творчества», имеют право в течение 15 (пятнадцати)
рабочих дней подать жалобу Ревизору ОО «Открытая лаборатория технического творчества», который обязан рассмотреть ее в
срок, установленный законодательством. До принятия решения по жалобе лицо, подавшее ее, считается членом ОО «Открытая
лаборатория технического творчества».

Члены ОО «Открытая лаборатория технического творчества» не отвечают по обязательствам ОО «Открытая лаборатория
технического творчества», членами которого они являются.
\end{numberedpars}

\section{Высший и выборные органы ОО «Открытая лаборатория технического творчества», их компетенция}

\begin{numberedpars}
ОО «Открытая лаборатория технического творчества» является цельным общественным объединением граждан, не имеющим в своем
составе организационных структур.

Структуру ОО «Открытая лаборатория технического творчества» составляют:
\begin{numberedsubpars}
высший орган ОО «Открытая лаборатория технического творчества» "--- Конференция ОО «Открытая лаборатория технического
творчества» (далее "--- Конференция);

руководящий орган "--- Правление ОО «Открытая лаборатория технического творчества» (далее "--- Правление), из числа членов
которого избирается Председатель Правления (далее "--- Председатель);

контрольно-ревизионный орган "--- Ревизор ОО «Открытая лаборатория технического творчества» (далее "--- Ревизор).
\end{numberedsubpars}

Заседания Конференции проводятся по мере необходимости, но не реже, чем один раз в календарный год.

Внеочередные заседания Конференции созываются:
\begin{numberedsubpars}
по решению Правления ОО «Открытая лаборатория технического творчества»;

по требованию Ревизора ОО «Открытая лаборатория технического творчества»;

по требованию 50\% и более членов ОО «Открытая лаборатория технического творчества».
\end{numberedsubpars}

К исключительной компетенции Конференции относится:
\begin{numberedsubpars}
определение основных направлений деятельности ОО «Открытая лаборатория технического творчества»;

утверждение названия ОО «Открытая лаборатория технического творчества»;

утверждение Устава ОО «Открытая лаборатория технического творчества», внесение любых изменений и дополнений в Устав ОО
«Открытая лаборатория технического творчества»;

ежегодное избрание членов Правления, избрание Ревизора ОО «Открытая лаборатория технического творчества» и досрочное
прекращение их полномочий;

принятие решения о реорганизации или ликвидации ОО «Открытая лаборатория технического творчества»;

решение иных вопросов, предусмотренных законодательством и настоящим Уставом.
\end{numberedsubpars}

Вопросы, отнесенные к исключительной компетенции Конференции, не могут быть переданы на решение Правлению либо
Председателю Правления ОО «Открытая лаборатория технического творчества».

Конференция признается правомочной (имеет кворум), если на ней присутствуют более чем две трети делегатов ОО «Открытая
лаборатория технического творчества». В случае отсутствия установленного кворума ежегодная Конференция должна, а
внеочередная "--- может быть проведена повторно с той же повесткой дня. Повторная Конференция имеет кворум, если на ней
присутствуют более половины делегатов ОО «Открытая лаборатория технического творчества».

Решения Конференции принимаются простым большинством голосов делегатов, принявших участие в Конференции. При равенстве
голосов решающим является голос Председателя Конференции, который голосует последним.

Конференция решает вопросы на своих заседаниях. Подготовку, созыв и проведение Конференции осуществляет Правление ОО
«Открытая лаборатория технического творчества».

Решение о созыве и проведении очередной Конференции должно быть принято Правлением не позднее 30 дней после окончания
отчётного периода.

Внеочередная Конференция должна быть проведена не позднее 30 дней с момента принятия Правлением решения о созыве и
проведении этой Конференции. Правление не позднее пятнадцати дней с момента получения требования о проведении
внеочередной Конференции обязано рассмотреть данное требование и принять решение о созыве и проведении внеочередной
Конференции либо мотивированное решение об отказе в ее созыве и проведении. Решение Правления о созыве и проведении
внеочередной Конференции либо мотивированное решение об отказе в ее созыве и проведении направляются лицам, требующим ее
созыва, посредством почтовой связи или иным способом, обеспечивающим подлинность передаваемых и принимаемых сообщений и
их документальное подтверждение, не позднее пяти дней с момента принятия такого решения.

\shortpage{}Правление в двухмесячный срок принимает решение о проведении Конференции, в котором должны быть определены:
дата, время и место (с указанием адреса) проведения Конференции;
повестка дня Конференции;

Решение о проведении Конференции может содержать и иные сведения, указание которых целесообразно в каждом конкретном
случае.

Лица, имеющие право на участие в Конференции, извещаются о принятом решении о проведении Конференции Правлением не менее
чем за десять дней до даты её проведения. Каждый член ОО «Открытая лаборатория технического творчества» извещается по
адресу, указанному в списке членов ОО «Открытая лаборатория технического творчества» посредством почтовой связи или иным
способом, обеспечивающим подлинность передаваемых и принимаемых сообщений и их документальное подтверждение.

Извещение о проведении повторной Конференции должно быть направлено не менее чем за десять дней до даты её проведения.

Делегатами, принявшими участие в Конференции, считаются зарегистрировавшиеся для участия в ней. Регистрация делегатов,
имеющих право на участие в Конференции, осуществляется при предъявлении ими документов, удостоверяющих личность, и при
наличии делегата в списке делегатов Конференции ОО «Открытая лаборатория технического творчества», который ведет
Правление ОО «Открытая лаборатория технического творчества». После регистрации делегатов определяется правомочность
(наличие кворума) этой Конференции. Делегаты, не прошедшие регистрацию, не вправе принимать участие в голосовании на
Конференции ОО «Открытая лаборатория технического творчества».

Предложение в повестку дня Конференции должно содержать имя физического лица и формулировку каждого из предлагаемых в
повестку дня вопросов, а также его подпись. Предложение в повестку дня о выдвижении кандидатов в избираемые (образуемые)
органы ОО «Открытая лаборатория технического творчества» должно также содержать имя каждого предлагаемого кандидата,
наименование органа ОО «Открытая лаборатория технического творчества», для избрания в который он предлагается.

Повестка дня Конференции формируется Председателем по своему усмотрению, а также на основании предложений членов ОО
«Открытая лаборатория технического творчества». 

Председатель не позднее десяти дней после окончания срока, установленного для поступления предложений в повестку дня,
обязан рассмотреть эти предложения и принять решение об их учете либо об отказе в их принятии в случаях, предусмотренных
законодательством. В случае отказа в принятии предложений Председатель должен направить лицу, внесшему эти предложения,
своё мотивированное решение не позднее пяти дней с даты принятия такого решения.

Конференция проводится в очной форме, которая предусматривает совместное присутствие лиц, имеющих право на участие в
этом заседании, при обсуждении вопросов повестки дня Конференции и принятии решений по ним.

Форма голосования при приеме решений Конференции определяется самой Конференцией.

Решения, принятые Конференцией, оглашаются на этой Конференции либо доводятся до сведения членов ОО «Открытая
лаборатория технического творчества» не позднее десяти дней после даты подписания протокола этой Конференции посредством
направления в адрес членов ОО «Открытая лаборатория технического творчества» копии указанного протокола. Члены ОО
«Открытая лаборатория технического творчества» имеют право ознакомиться с протоколом Конференции по юридическому адресу
ОО «Открытая лаборатория технического творчества».

Правление является руководящим выборным органом ОО «Открытая лаборатория технического творчества». Правление
осуществляет руководство деятельностью ОО «Открытая лаборатория технического творчества» в период между заседаниями
Конференции.

Правление состоит из пяти членов ОО «Открытая лаборатория технического творчества», которые избираются ежегодно путем
открытого голосования делегатов Конференции ОО «Открытая лаборатория технического творчества». В состав Правления могут
быть избраны только члены ОО «Открытая лаборатория технического творчества», достигшие восемнадцатилетнего возраста. Не
допускается одновременное занятие членом ОО «Открытая лаборатория технического творчества» должности Ревизора и члена
Правления ОО «Открытая лаборатория технического творчества».

\pagebreak{}Правление:
\begin{numberedsubpars}
принимает решение о приобретении членства в ОО «Открытая лаборатория технического творчества» и об исключении из членов
ОО «Открытая лаборатория технического творчества»;

в период между заседаниями Конференции вносит в Устав ОО «Открытая лаборатория технического творчества» изменения и
(или) дополнения, связанные с изменением его юридического адреса либо обусловленные изменениями в законодательстве
Республики Беларусь;

ведет список членов ОО «Открытая лаборатория технического творчества»;

принимает решение о создании других юридических лиц, в том числе коммерческих организаций, а также решение об участии ОО
«Открытая лаборатория технического творчества» в других юридических лицах;

утверждает регламент проведения Конференции;

определяет норму делегирования и порядок избрания делегатов для участия в Конференции;

принимает иные решения, обязательные для всех органов и членов ОО «Открытая лаборатория технического творчества», не
отнесенные к исключительной компетенции Конференции.
\end{numberedsubpars}

Правление проводит заседания по мере необходимости, но не реже трёх раз в год. Извещение о проведении заседания
Правления должно быть направлено Председателем Правления всем членам Правления не менее чем за десять дней до даты его
проведения.

Председатель Правления формирует вопросы повестки дня на основании предложений членов Правления, направивших свои
предложения о рассмотрении отдельных вопросов деятельности ОО «Открытая лаборатория технического творчества».

Заседание Правления считается правомочным, если на нем присутствует не менее половины членов Правления. Решение
Правления считается принятым, если за него проголосовало более половины присутствующих членов. При равном количестве
голосов голос Председателя Правления имеет решающее значение.

Заседание Правления ведет Председатель Правления. Ведение протокола Правления обеспечивает Секретарь, избираемый на
заседании Правления. Протокол составляется в двух экземплярах и подписывается Председателем и Секретарем. 

Решения Правления могут приниматься открытым голосованием либо тайным голосованием бюллетенями.

Решения, принятые Правлением, оформляются протоколом, оглашаются на заседании Правления и доводятся до сведения членов
ОО «Открытая лаборатория технического творчества» не позднее десяти рабочих дней после даты подписания протокола
посредством направления в адрес членов ОО «Открытая лаборатория технического творчества» копии указанного протокола.
Члены ОО «Открытая лаборатория технического творчества» имеют право ознакомиться с протоколом Правления по юридическому
адресу ОО «Открытая лаборатория технического творчества».

Внеочередные заседания Правления созываются по решению Председателя или по инициативе большинства его членов.
Решения Правления могут быть обжалованы любым членом ОО «Открытая лаборатория технического творчества» Ревизору в
течение 15 (пятнадцати) рабочих дней со дня уведомления члена ОО «Открытая лаборатория технического творчества» о
принятом на заседании Правления решении. Ревизор обязан рассмотреть указанную жалобу в срок, установленный
законодательством. 

Председатель Правления:
\begin{numberedsubpars}
осуществляет текущее руководство деятельностью ОО «Открытая лаборатория технического творчества»;

в пределах своей компетенции без доверенности действует от имени ОО «Открытая лаборатория технического творчества», в
том числе представляет интересы ОО «Открытая лаборатория технического творчества» и подписывает необходимые документы от
имени ОО «Открытая лаборатория технического творчества»;

нанимает и увольняет штатных работников в соответствии с законодательством и условиями контрактов (договоров), в том
числе гражданско-правовых; утверждает штатное расписание ОО «Открытая лаборатория технического творчества»;

представляет ОО «Открытая лаборатория технического творчества» без доверенности в отношениях с государственными органами
Республики Беларусь и других государств, юридическими и физическими лицами;

в пределах своих полномочий, определенных в договоре (контракте), приобретает от имени ОО «Открытая лаборатория
технического творчества» и распоряжается имуществом, в том числе средствами ОО «Открытая лаборатория технического
творчества»;

открывает в банках текущий (расчетный), валютный и иные счета предусмотренные законодательством, закрывает счета ОО
«Открытая лаборатория технического творчества», распоряжается денежными средствами ОО «Открытая лаборатория технического
творчества» на счетах в пределах, установленных договором (контрактом), настоящим Уставом и законодательством;

выдает доверенности;

организует подготовку, созыв и проведение заседания Правления ОО «Открытая лаборатория технического творчества»;

председательствует на заседаниях Правления ОО «Открытая лаборатория технического творчества»;

заключает от имени ОО «Открытая лаборатория технического творчества» договоры и обеспечивает их выполнение;

принимает иные решения, обязательные для всех органов и членов ОО «Открытая лаборатория технического творчества», не
отнесенные к исключительной компетенции Конференции и Правления ОО «Открытая лаборатория технического творчества».
\end{numberedsubpars}

Председатель в пределах своих полномочий издаёт приказы и даёт указания, обязательные для всех штатных работников ОО
«Открытая лаборатория технического творчества».

Решения Председателя Правления могут быть обжалованы любым членом ОО «Открытая лаборатория технического творчества»
Ревизору в течение 15 (пятнадцати) рабочих дней со дня, когда член ОО «Открытая лаборатория технического творчества»
узнал о принятом Председателем решении. Ревизор обязан рассмотреть указанную жалобу в срок, установленный
законодательством.

Ревизор ОО «Открытая лаборатория технического творчества» осуществляет внутреннюю проверку финансово-хозяйственной
деятельности ОО «Открытая лаборатория технического творчества», а также внутренний контроль за соответствием
деятельности ОО «Открытая лаборатория технического творчества» законодательству и его Уставу.

Ревизор подотчетен Конференции.

Ревизор ОО «Открытая лаборатория технического творчества»:
\begin{numberedsubpars}
контролирует и осуществляет ревизии деятельности выборных органов и должностных лиц ОО «Открытая лаборатория
технического творчества»;

контролирует учет материальных ценностей и денежных средств ОО «Открытая лаборатория технического творчества»;

проверяет организацию делопроизводства ОО «Открытая лаборатория технического творчества», сроки, законность и
обоснованность ответов на заявления, жалобы и письма членов ОО «Открытая лаборатория технического творчества»;

по мере необходимости, но не реже одного раза в год проводит ревизию деятельности ОО «Открытая лаборатория технического
творчества» и оформляет результат проверки заключением;

проводит ревизию деятельности ОО «Открытая лаборатория технического творчества» по решению Конференции в установленные
им сроки;

по своей инициативе проводит ревизию деятельности ОО «Открытая лаборатория технического творчества».
\end{numberedsubpars}

Продолжительность ревизии или проверки не должна превышать 30 (тридцати) дней.

\pagebreak{}Ревизор ОО «Открытая лаборатория технического творчества» в случае выявления нарушений обязан:
\begin{numberedsubpars}
представить заключение ревизии или проверки либо отдельные их выводы и предложения Правлению, которое в соответствии с
его компетенцией в двухнедельный срок обязано принять меры по устранению допущенных нарушений;

потребовать созыва внеочередной Конференции, если по выявленным в ходе ревизии или проверки фактам нарушений решение
может быть принято только этой Конференцией.
\end{numberedsubpars}

Ревизор может присутствовать на заседаниях Правления с правом совещательного голоса.

Результаты проверок Ревизора оформляются актами и (или) справками.

Компетенция Ревизора по вопросам, не предусмотренным настоящим Уставом, определяются решением Конференции.
\end{numberedpars}

\section{Имущество и средства ОО «Открытая лаборатория технического творчества»}

\begin{numberedpars}
ОО «Открытая лаборатория технического творчества» может иметь в собственности любое имущество, необходимое для
материального обеспечения его деятельности, предусмотренной настоящим Уставом, за исключением объектов, которые согласно
законодательству могут находиться только в собственности государства.

Собственником имущества ОО «Открытая лаборатория технического творчества» является ОО «Открытая лаборатория технического
творчества». Члены ОО «Открытая лаборатория технического творчества» не имеют права собственности на долю в его
имуществе. Имущество, переданное в собственность ОО «Открытая лаборатория технического творчества» его членами, другими
лицами, является собственностью данного объединения.

\pagebreak{}Денежные средства ОО «Открытая лаборатория технического творчества» формируются из:
\begin{numberedsubpars}
вступительных и членских взносов;

поступлений от проводимых в уставных целях лекций, выставок, спортивных и других мероприятий;

доходов от предпринимательской деятельности, осуществляемой в установленном настоящим Уставом и законодательством
порядке;

добровольных пожертвований;

иных источников, не запрещенных законодательством.
\end{numberedsubpars}

Денежные средства ОО «Открытая лаборатория технического творчества» расходуются на:
\begin{numberedsubpars}
выполнение задач, стоящих перед ОО «Открытая лаборатория технического творчества»;

финансирование планов, программ, проектов исследований в соответствии с уставными целями и задачами ОО «Открытая
лаборатория технического творчества»;

обеспечение работы выборных органов ОО «Открытая лаборатория технического творчества»;

выплату заработной платы, улучшение социально-бытовых условий штатных сотрудников исполнительного аппарата и членов ОО
«Открытая лаборатория технического творчества», их премирование и поощрение в соответствии с настоящим Уставом и
законодательством;

создание юридических лиц, деятельность которых направлена на решение уставных задач;

развитие материально-технической базы ОО «Открытая лаборатория технического творчества»;

благотворительную деятельность;

на иные цели, предусмотренные настоящим Уставом, законодательством.
\end{numberedsubpars}

Денежные средства и иное имущество ОО «Открытая лаборатория технического творчества» не могут перераспределяться между
членами этого ОО «Открытая лаборатория технического творчества» и используются только для выполнения уставных целей и
задач.

ОО «Открытая лаборатория технического творчества» не отвечает по обязательствам своих членов. Члены ОО «Открытая
лаборатория технического творчества» не отвечают по обязательствам ОО «Открытая лаборатория технического творчества»,
членами которого они являются.

Права владения, пользования, распоряжения имуществом, находящимся в собственности ОО «Открытая лаборатория технического
творчества», осуществляет Председатель Правления ОО «Открытая лаборатория технического творчества» в соответствии с
законодательством и настоящим Уставом в пределах, установленных Правлением. 
\end{numberedpars}

\section{Права ОО «Открытая лаборатория технического творчества»}

\begin{numberedpars}
ОО «Открытая лаборатория технического творчества» не отвечает по обязательствам своих членов. 

ОО «Открытая лаборатория технического творчества» имеет право:
\begin{numberedsubpars}
осуществлять деятельность, направленную на достижение уставных целей;

беспрепятственно получать и распространять информацию, имеющую отношение к деятельности ОО «Открытая лаборатория
технического творчества»;

пользоваться государственными средствами массовой информации в порядке, установленном законодательством;

учреждать собственные средства массовой информации и осуществлять издательскую деятельность в порядке, установленном
законодательством;

защищать права и законные интересы, а также представлять законные интересы своих членов в государственных органах и иных
организациях;

поддерживать связи с другими общественными объединениями, союзами и ассоциациями;

создавать союзы;

осуществлять в установленном порядке предпринимательскую деятельность лишь постольку, поскольку она необходима для
уставных целей, ради которых оно создано, соответствует этим целям и отвечает предмету деятельности. Указанная
деятельность может осуществляться ОО «Открытая лаборатория технического творчества» только посредством образования
коммерческих организаций и (или) участия в них;

осуществлять иные права в соответствии с настоящим Уставом и действующим законодательством.
\end{numberedsubpars}
\end{numberedpars}

\section{Реорганизация и ликвидация ОО «Открытая лаборатория технического творчества»}

\begin{numberedpars}
Прекращение деятельности ОО «Открытая лаборатория технического творчества» происходит в случае его реорганизации либо
ликвидации.

Реорганизация или ликвидация ОО «Открытая лаборатория технического творчества» осуществляется по решению Конференции ОО
«Открытая лаборатория технического творчества», принятому не менее 2/3 голосов всех членов ОО «Открытая лаборатория
технического творчества» либо их делегатов, в соответствии с законодательством Республики Беларусь.

Ликвидация ОО «Открытая лаборатория технического творчества» также осуществляется по решению суда в случаях и порядке,
предусмотренных законодательством Республики Беларусь.

Ликвидационная комиссия создается органом, принявшим решение о ликвидации.

В случае ликвидации ОО «Открытая лаборатория технического творчества», принадлежащее ему имущество и денежные средства,
оставшиеся после удовлетворения требований кредиторов, используются по решению ликвидационной комиссии на цели,
предусмотренные настоящим уставом.
\end{numberedpars}

\end{document}
