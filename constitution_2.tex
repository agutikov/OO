\documentclass[a4paper,fontsize=14pt,titlepage]{scrartcl}

\usepackage[T2A]{fontenc}
\usepackage[utf8]{inputenc}
\usepackage[russian]{babel}
\usepackage{indentfirst}
\usepackage{graphicx}
\usepackage[left=2cm,top=2cm,right=1.7cm,bottom=2cm,nohead,nofoot]{geometry}
\usepackage[font={small}]{caption}
\usepackage{wrapfig}
\usepackage{enumitem}

\setlist{nolistsep}
%\setlist[1]{\labelindent=\parindent}
%\setlist[itemize]{leftmargin=*}
\setlist[itemize,1]{label=---}

\setkomafont{section}{\normalfont\bfseries\centering}
\setkomafont{paragraph}{\normalfont}
\setkomafont{subparagraph}{\normalfont}
\setkomafont{title}{\rmfamily \huge}

\newcommand{\longpage}{\enlargethispage{\baselineskip}}
\newcommand{\shortpage}{\enlargethispage{-\baselineskip}}

% deep magic follows, don't touch it!
\makeatletter
\@newctr{paragraph}[section]
\@newctr{subparagraph}[paragraph]
\newenvironment{numberedpars}{%
  \addtocounter{secnumdepth}{1}
  \renewcommand\theparagraph{\arabic{section}.\arabic{paragraph}.}
  \renewcommand\@seccntformat[1]
  {\expandafter\ifx\csname##1\endcsname\paragraph\csname
  the##1\endcsname\else\csname the##1\endcsname\fi}
  \def\paragraph{\par\addvspace{3.25ex \@plus1ex \@minus.2ex}{\raggedsection\normalfont\sectfont\nobreak\size@paragraph\refstepcounter{paragraph}\@seccntformat{paragraph}}\,}
  \let\old@par=\par
  \def\new@par{\let\par=\old@par\paragraph{}\let\par=\new@par}
  \let\par=\new@par
  \par
}{
  \addtocounter{secnumdepth}{-1}
}
\newenvironment{numberedsubpars}{%
  \addtocounter{secnumdepth}{1}
  \renewcommand\thesubparagraph{\arabic{section}.\arabic{paragraph}.\arabic{subparagraph}.}
  \renewcommand\@seccntformat[1]
  {\expandafter\ifx\csname##1\endcsname\subparagraph\csname
  the##1\endcsname\else\csname the##1\endcsname\fi}
  \def\subparagraph{\par\addvspace{3.25ex \@plus1ex \@minus .2ex}{\raggedsection\normalfont\sectfont\nobreak\size@paragraph\refstepcounter{subparagraph}\@seccntformat{subparagraph}}\,}
  \let\old@@par=\par
  \def\new@@par{\let\par=\old@par\subparagraph{}\let\par=\new@@par}
  \let\par=\new@@par
  \par
}{
  \addtocounter{secnumdepth}{-1}
  \let\par=\old@@par
}

\def\emph{\textbf}

\renewcommand\thesection{\arabic{section}.}
\let\@@@section=\section
\renewcommand\section[1]{\@@@section{\MakeUppercase{#1}}}
% end of magic

\makeatother

\sloppy

\begin{document}

\setkomafont{title}{\bfseries}

\date{\large г. Минск\\2013}

\titlehead{\raggedleft \begin{minipage}{15em}%
Принят на\\%
Учредительном Cобрании\\%
общественного объединения\\%
«Открытая лаборатория\\%
технического творчества»\\%
25 июля 2013 года%%
\end{minipage}}


\title{\Large%
УСТАВ\\%
\textbf{\large ОБЩЕСТВЕННОГО ОБЪЕДИНЕНИЯ}\\%
«Открытая лаборатория технического творчества»\\[1ex]%
(ОО «Открытая лаборатория\\%
технического творчества»)\\[4ex]%
СТАТУТ\\%
\textbf{\large ГРАМАДСКАГА АБ’ЯДНАННЯ}\\%
«Адкрытая лабараторыя тэхнічнай творчасці»\\[1ex]%
(ГА «Адкрытая лабараторыя\\%
тэхнічнай творчасці»)%
}
\maketitle


\section{ОБЩИЕ ПОЛОЖЕНИЯ}


\begin{numberedpars}
Общественное объединение «Открытая лаборатория технического творчества»
(далее "--- ОО «Открытая лаборатория технического творчества»)
является добровольным объединением граждан,
объединившихся на основе общности интересов в установленном законодательством порядке.
\end{numberedpars}
\begin{numberedpars}
Полное название объединения:
\end{numberedpars}
\begin{itemize}
\item на русском языке: \emph{Общественное объединение «Открытая лаборатория технического творчества»};
\item на белорусском языке: \emph{Грамадскае аб’яднанне «Адкрытая лабараторыя тэхнічнай творчасці»}.
\end{itemize}
Сокращенное название объединения:
\begin{itemize}
\item на русском языке: \emph{ОО «Открытая лаборатория технического творчества»};
\item на белорусском языке: \emph{ГА «Адкрытая лабараторыя тэхнічнай творчасці»}.
\end{itemize}
\begin{numberedpars}
ОО «Открытая лаборатория технического творчества»
имеет статус местного общественного объединения.
Территория распространения деятельности ОО «Открытая лаборатория технического творчества» "--- г. Минск.
\end{numberedpars}
\begin{numberedpars}
ОО «Открытая лаборатория технического творчества»
создается и действует в соответствии с законодательством Республики Беларусь
и настоящим Уставом.
\end{numberedpars}
\begin{numberedpars}
ОО «Открытая лаборатория технического творчества»
с момента регистрации является юридическим лицом,
несет самостоятельную ответственность по своим обязательствам,
имеет обособленное имущество,
самостоятельный баланс и счета в банках,
от своего имени выступает во взаимоотношениях с юридическими и физическими лицами,
может быть истцом и ответчиком в судах,
имеет печать,
бланки со своим наименованием,
может иметь собственную символику,
которая подлежит регистрации в установленном порядке.
\end{numberedpars}
\begin{numberedpars}
ОО «Открытая лаборатория технического творчества»
может вступать в международные общественные объединения,
созданные на территории иностранных государств,
союзы, участвовать в их создании,
поддерживать прямые международные контакты и связи в соответствии с законодательством.
\end{numberedpars}
\begin{numberedpars}
Юридический адрес ОО «Открытая лаборатория технического творчества»: .
\end{numberedpars}



\newpage\section{ЦЕЛЬ, ЗАДАЧИ, ПРЕДМЕТ И МЕТОДЫ ДЕЯТЕЛЬНОСТИ}


\begin{numberedpars}
Целями ОО «Открытая лаборатория технического творчества» являются развитие современных информационных
технологий, науки, техники и электроники, повышение уровня знаний и навыков населения в данной области.
\end{numberedpars}
\begin{numberedpars}
Задачами ОО «Открытая лаборатория технического творчества» являются:
\end{numberedpars}
\begin{itemize}
\item поддержка и популяризация технического творчества;
\item cоздание и обеспечение функционирования центров технического творчества;
\item создание благоприятной среды для обмена знаниями и опытом, общения и совместной работы для участников проектов,
связанных с техническим творчеством;
\item распространение культуры и принципов свободного программного и аппаратного обеспечения.
\end{itemize}
\begin{numberedpars}
Предметом деятельности ОО «Открытая лаборатория технического творчества» являются просветительская, образовательная деятельность.
\end{numberedpars}
\begin{numberedpars}
Для достижения своей цели и решения задач ОО «Открытая лаборатория технического творчества» в порядке,
установленном законодательством, использует следующие методы деятельности:
\end{numberedpars}
\begin{itemize}
\item распространение информации о деятельности сообщества;
\item проведение обучающих семинаров и лекций;
\item участие в организации выставок, конференций, круглых столов и иных аналогичных мероприятий, представляющих интерес для
членов технического сообщества;
\item проведение конкурсов и соревнований;
\item создание и обеспечение работы центров технического творчества, клубов радиолюбителей-конструкторов, программистов, и
подобных;
\item содействие и поддержка в реализации оригинальных проектов участников сообщества;
\item инициирование и участие в совместных проектах с учебными(государственными и негосударственными) учереждениями,
научно-образовательными центрами и им подобными;
\item создание и публикация экспертных исследований, аналитических обзоров,
информационных и обучающих материалов, рекомендаций, касающихся актуальных вопросов социальных,
экономических, правовых, технологических аспектов разработки и сопровождения свободного программного и аппаратного обеспечения,
открытых форматов данных;
\item в установленном законодательством порядке вступление в международные общественные (неправительственные) объединения и
союзы, установление и поддержка прямых международных контактов и связей, заключение для этих целей соответствующих
соглашений;
\item получение финансовой и иной поддержки из различных не запрещённых законодательством Республики Беларусь источников;
\item иные методы, не противоречащие законодательству Республики Беларусь.
\end{itemize}
Деятельность, на осуществление которой требуется специальное разрешение (лицензия),
осуществляется только после получения необходимого разрешения (лицензии) в установленном порядке.




\newpage\section{ЧЛЕНСТВО, ПРАВА И ОБЯЗАННОСТИ ЧЛЕНОВ ОО «Открытая лаборатория технического творчества»}


\begin{numberedpars}
Членами ОО «Открытая лаборатория технического творчества» могут быть, в соответствии с законодательством,
граждане Республики Беларусь, а также иностранные граждане и лица без гражданства,
достигшие возраста 18 лет, признающие Устав и желающие участвовать в достижении его цели.
Лица, не достигшие 16 лет, могут вступать в ОО «Открытая лаборатория технического творчества» с письменного
согласия своих законных представителей.
\end{numberedpars}
\begin{numberedpars}
Прием в члены ОО «Открытая лаборатория технического творчества» осуществляется решением
Правления ОО «Открытая лаборатория технического творчества» по заявлению вступающего.
Прекращение членства может быть осуществлено путем выхода из членов объединения либо в случае исключения из числа членов
ОО «Открытая лаборатория технического творчества».
\end{numberedpars}
\begin{numberedpars}
Размер вступительного и членских взносов, порядок их уплаты определяется Правлением ОО «Открытая лаборатория технического творчества».
Правление ОО «Открытая лаборатория технического творчества» может принять решение об освобождении
отдельных граждан от уплаты вступительного и членских взносов, для отдельных членов
ОО «Открытая лаборатория технического творчества» размер вступительного и членских взносов может быть снижен.
\end{numberedpars}
\begin{numberedpars}
Выход из членов ОО «Открытая лаборатория технического творчества» осуществляется путем подачи заявления
в Правление ОО «Открытая лаборатория технического творчества», при этом членство считается прекращенным с даты, указанной в заявлении.
\end{numberedpars}
\begin{numberedpars}
Решение об исключении из числа членов ОО «Открытая лаборатория технического творчества» может быть принято
Правлением ОО «Открытая лаборатория технического творчества» в случае грубого нарушения Устава членом объединения.
Членство считается прекращенным с даты, указанной в решении Правления ОО «Открытая лаборатория технического творчества» об
исключении из числа членов ОО «Открытая лаборатория технического творчества».
\end{numberedpars}

\begin{numberedpars}
Член ОО «Открытая лаборатория технического творчества» имеет право:
\end{numberedpars}
\begin{itemize}
\item принимать участие в мероприятиях ОО «Открытая лаборатория технического творчества», в заседаниях высшего органа;
\item избирать и выдвигать свою кандидатуру на выборах в выборные органы ОО «Открытая лаборатория технического творчества»;
\item получать от органов и должностных лиц ОО «Открытая лаборатория технического творчества» информацию,
касающуюся деятельности ОО «Открытая лаборатория технического творчества»;
\item обжаловать решения органов и должностных лиц ОО «Открытая лаборатория технического творчества» в порядке,
предусмотренном Уставом и законодательством;
\item вносить предложения относительно деятельности ОО «Открытая лаборатория технического творчества»
на рассмотрение выборных органов объединения;
\item вносить добровольные пожертвования для поддержки деятельности ОО «Открытая лаборатория технического творчества»;
\item свободного выхода из членов ОО «Открытая лаборатория технического творчества».
\end{itemize}

\begin{numberedpars}
Член ОО «Открытая лаборатория технического творчества» обязан:
\end{numberedpars}
\begin{itemize}
\item выполнять требования Устава;
\item участвовать в работе по выполнению целей и задач ОО «Открытая лаборатория технического творчества»;
\item не совершать действий, наносящих материальный ущерб или причиняющих вред
деловой репутации ОО «Открытая лаборатория технического творчества».
\end{itemize}
\begin{numberedpars}
Учет членов в ОО «Открытая лаборатория технического творчества» осуществляется Правлением ОО «Открытая лаборатория технического творчества»
путем ведения списка членов, который редактируется по мере необходимости и обновляется по мере вступления и выбытия членов.
\end{numberedpars}

\newpage\section{ВЫСШИЙ И ВЫБОРНЫЕ ОРГАНЫ ОО «Открытая лаборатория технического творчества»}

\begin{numberedpars}
ОО «Открытая лаборатория технического творчества» является цельным общественным объединением, не имеющим в своём
составе организационных структур.
\end{numberedpars}
\begin{numberedpars}
Высшим органом ОО «Открытая лаборатория технического творчества» является Общее Собрание ОО «Открытая лаборатория технического творчества»
(далее "--- Общее Собрание), которое созывается Правлением ОО «Открытая лаборатория технического творчества» по мере необходимости,
но не реже одного раза в год.
Общее Собрание может быть созвано по требованию Правления ОО «Открытая лаборатория технического творчества»
либо Ревизора ОО «Открытая лаборатория технического творчества», либо по инициативе не менее одной пятой части всех членов
ОО «Открытая лаборатория технического творчества».
Время, место проведения, повестка дня Общего Собрания определяются Правлением ОО «Открытая лаборатория технического творчества»
и сообщаются членам объединения не позднее, чем за 5 (пять) дней до дня Общего Собрания.
Общее Собрание считается правомочным, если в нем участвует не менее половины членов ОО «Открытая лаборатория технического творчества».
Форма и порядок голосования устанавливается Собщим Собранием.
\end{numberedpars}
\begin{numberedpars}
К компетенции Общего Собрания относится:
\end{numberedpars}
\begin{itemize}
\item определение основных направлений и форм деятельности ОО «Открытая лаборатория технического творчества»;
утверждение названия, принятие Устава ОО «Открытая лаборатория технического творчества», внесение в него изменений и/или дополнений,
кроме случаев, когда в соответствии с законодательством и Уставом изменения и/или дополнения в Устав могут быть внесены Правлением;
\item избрание сроком на 1 (один) год Правления в составе не менее 5 (пяти) человек, Председателя Правления,
Заместителя Председателя Правления, Ревизора;
\item заслушивание отчетов Правления, Ревизора;
\item принятие решения о реорганизации или ликвидации ОО «Открытая лаборатория технического творчества».
\end{itemize}
Общее Собрание может принять к рассмотрению любой другой вопрос деятельности
ОО «Открытая лаборатория технического творчества».
\begin{numberedpars}
В период между Общими Собраниями деятельностью ОО «Открытая лаборатория технического творчества» руководит
Правление, являющееся руководящим органом ОО «Открытая лаборатория технического творчества».
Правление избирается сроком на 1 (один) год.
Заседания Правления созываются Председателем Правления по мере необходимости, но не реже одного раза в 3 (три) месяца.
\end{numberedpars}
\begin{numberedpars}
Правление правомочно, если на его заседании присутствует не менее половины членов Правления.
\end{numberedpars}
\begin{numberedpars}
Правление ОО «Открытая лаборатория технического творчества»:
\end{numberedpars}
\begin{itemize}
\item организует деятельность ОО «Открытая лаборатория технического творчества», исходя из его цели, задач и методов;
\item созывает и организует работу Общего Собрания;
\item утверждает отчеты о доходах и расходах ОО «Открытая лаборатория технического творчества» и сметы расходов на
будущий период;
\item принимает решения об участии в создании или вступлении в союзы и ассоциации;
\item принимает решения о создании и ликвидации юридических лиц, созданных при
участии ОО «Открытая лаборатория технического творчества» в соответствии с законодательством, утверждает их
уставы и руководителей;
\item вносит изменения и/или дополнения в Устав, связанные с переменой юридического адреса
либо обусловленные изменениями в законодательстве;
\item утверждает штатное расписание и должностные оклады, нанимает и увольняет
штатных работников;
\item принимает решения о приобретении, распоряжении и отчуждении имущества
ОО «Открытая лаборатория технического творчества»;
\item заслушивает отчеты Председателя Правления, руководителей созданных ОО «Открытая лаборатория технического творчества» юридических лиц;
\item утверждает образцы печати, штампов, символики ОО «Открытая лаборатория технического творчества»;
\item решает иные вопросы деятельности ОО «Открытая лаборатория технического творчества», не относящиеся согласно
Уставу к компетенции других органов ОО «Открытая лаборатория технического творчества».
\end{itemize}
\begin{numberedpars}
Председатель Правления является руководителем юридического лица.
Председатель Правления входит в состав Правления по должности.
\end{numberedpars}
\begin{numberedpars}
Председатель Правления:
\end{numberedpars}
\begin{itemize}
\item созывает заседания Правления, определяет время и место их проведения,
оповещает всех членов Правления не позднее чем за 2 (два) дня до заседания;
\item председательствует на заседаниях Правления;
\item осуществляет контроль за выполнением решений Правления;
\item обеспечивает выполнение решений Общего Собрания и Правления;
\item без доверенности действует от имени ОО «Открытая лаборатория технического творчества», представляет его интересы;
\item принимает решения по вопросам, которые не отнесены к компетенции Общего Собрания и Правления;
\item принимает решения и издает распоряжения по текущим вопросам деятельности ОО «Открытая лаборатория технического творчества»;
\item заключает договоры от имени ОО «Открытая лаборатория технического творчества», выдает доверенности;
\item открывает расчетные и другие счета в банках;
\item распоряжается имуществом и средствами ОО «Открытая лаборатория технического творчества» в пределах, устанавливаемых Правлением;
\item решает другие вопросы, связанные с деятельностью ОО «Открытая лаборатория технического творчества»
и не отнесенные к компетенции иных органов.
\end{itemize}
При отсутствии Председателя Правления все его обязанности исполняет Заместитель Председателя Правления.
\begin{numberedpars}
Для осуществления внутренней проверки финансово-хозяйственной деятельности,
а также внутреннего контроля за соответствием деятельности ОО «Открытая лаборатория технического творчества»
учредительным документам и законодательству Общее Собрание избирает Ревизора, подотчётного Общему Собранию.
\end{numberedpars}
\begin{numberedpars}
Ревизор ОО «Открытая лаборатория технического творчества»:
\end{numberedpars}
\begin{itemize}
\item контролирует деятельность ОО «Открытая лаборатория технического творчества»;
\item проверяет бухгалтерские счета и книги, просматривает документы ОО «Открытая лаборатория технического творчества» в любое время
в период действия своих полномочий;
\item проверяет обоснованность ответов на письма, жалобы, предложения членов ОО
«Открытая лаборатория технического творчества»;
\item проверяет организацию делопроизводства и отчетности ОО «Открытая лаборатория технического творчества».
\end{itemize}
\begin{numberedpars}
Ревизор проводит проверки по мере необходимости, но не реже раза в 1 (один) год.
Ревизор в случае необходимости вправе привлекать к своей работе специалистов для консультаций и участия в проведении ревизий.
\end{numberedpars}
\begin{numberedpars}
Решения коллегиальных органов ОО «Открытая лаборатория технического творчества» оформляются протоколами.
Председатель Правления издает приказы и распоряжения.
Ревизор оформляет проверки актами и справками.
\end{numberedpars}
\begin{numberedpars}
Решения коллегиальных органов ОО «Открытая лаборатория технического творчества» (и его организационных
структур) принимаются открытым голосованием простым большинством голосов
от числа присутствующих, если Уставом не оговорено иное.
\end{numberedpars}
\begin{numberedpars}
Решение Председателя может быть обжаловано Ревизору.
Ревизор в течение месяца обязан рассмотреть жалобу и вынеси по ней решение.
Решение Ревизора может быть обжаловано на Общем Собрании.
Ревизор в месячный срок обязан принять решение по существу рассматриваемой жалобы.
Ответ компетентного органа направляется в пятидневный срок заявителю.
Общее Собрание может принять к рассмотрению любой вопрос по жалобе на неправомерные действия
должностных лиц и выборных органов объединения.
\end{numberedpars}


\newpage\section{ФИНАНСОВЫЕ СРЕДСТВА И ИМУЩЕСТВО}


\begin{numberedpars}
ОО «Открытая лаборатория технического творчества» может иметь в собственности любое имущество, необходимое
для материального обеспечения уставной деятельности, за исключением объектов,
которые, согласно закону, могут находиться только в собственности государства.
\end{numberedpars}
\begin{numberedpars}
Денежные средства и имущество ОО «Открытая лаборатория технического творчества» формируются из:
\end{numberedpars}
\begin{itemize}
\item вступительных и членских взносов;
\item добровольных пожертвований физических и юридических лиц;
\item поступлений от проводимых в уставных целях мероприятий в соответствии с законодательством;
\item отчислений от созданных ОО «Открытая лаборатория технического творчества» предприятий;
\item других поступлений, не запрещенных законодательством.
\end{itemize}
Средства ОО «Открытая лаборатория технического творчества» расходуются на реализацию уставных цели и задач и
не могут перераспределяться между его членами.
\begin{numberedpars}
ОО «Открытая лаборатория технического творчества» несет ответственность по принятым на себя обязательствам
всем принадлежащим ему имуществом.
ОО «Открытая лаборатория технического творчества» не отвечает по обязательствам своих членов,
а они не отвечают по его обязательствам.
\end{numberedpars}
\begin{numberedpars}
В случае прекращения членства в ОО «Открытая лаборатория технического творчества» финансовые средства и имущество,
переданные его членами ОО «Открытая лаборатория технического творчества» в собственность безвозмездно, возврату не подлежат.
Материальные средства, переданные ОО «Открытая лаборатория технического творчества»
его членами во временное владение и пользование, возвращаются в соответствии с
условиями договоров, на основании которых это владение и пользование осуществлялось.
\end{numberedpars}


\newpage\section{ПРАВА ОО «Открытая лаборатория технического творчества»}


\begin{numberedpars}
Для достижения уставных цели и задач ОО «Открытая лаборатория технического творчества» вправе:
\end{numberedpars}
\begin{itemize}
\item осуществлять деятельность, направленную на достижение уставной цели;
\item беспрепятственно получать и распространять информацию, имеющую отношение к своей деятельности,
в порядке, установленном законодательством, учреждать собственные и пользоваться в установленном порядке государственными
средствами массовой информации, осуществлять издательскую деятельность;
\item представлять и защищать права и законные интересы своих членов в государственных органах и иных организациях;
\item поддерживать связи с другими некоммерческими организациями;
\item создавать некоммерческие организации, вступать в союзы (ассоциации);
\item иметь иные права, предусмотренные законодательными актами.
\end{itemize}
\begin{numberedpars}
ОО «Открытая лаборатория технического творчества» может осуществлять в установленном порядке предпринимательскую деятельность лишь постольку,
поскольку она необходима для уставных целей, ради которых оно создано, соответствует этим целям и отвечает предмету деятельности.
Такая деятельность может осуществляться только посредством образования коммерческих организаций и (или) участия в них.
\end{numberedpars}


\newpage\section{ПРЕКРАЩЕНИЕ ДЕЯТЕЛЬНОСТИ ОО «Открытая лаборатория технического творчества»}


\begin{numberedpars}
Прекращение деятельности ОО «Открытая лаборатория технического творчества» происходит путем его реорганизации или ликвидации.
\end{numberedpars}
\begin{numberedpars}
Реорганизация ОО «Открытая лаборатория технического творчества» производится по решению Общего Собрания.
Ликвидация ОО «Открытая лаборатория технического творчества» производится по решению Общего Собрания, либо по решению суда.
Решение о ликвидации направляется в регистрирующий орган и публикуется в периодическом печатном издании,
определенном актами законодательства.
Ликвидация производится ликвидационной комиссией, созданной органом, принявшим решение о ликвидации.
\end{numberedpars}
\begin{numberedpars}
При ликвидации средства и имущество ОО «Открытая лаборатория технического творчества»,
оставшиеся после полного удовлетворения всех имущественных требований кредиторов,
используются на цели, предусмотренные Уставом, если денежные средства и иное имущество объединения,
в соответствии с законодательными актами, не подлежат обращению в доход государства.
\end{numberedpars}


\end{document}
